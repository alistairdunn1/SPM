\section{Quick reference\label{sec:quick-reference}}
\subsection{Population command and subcommand syntax}\par
\defCom{model}{Define the spatial structure, population structure, annual cycle, and model years}\par
\defSub{nrows}{The number of rows $n_{rows}$ in the spatial structure}
\defSub{ncols}{The number of columns $n_{cols}$ in the spatial structure}
\defSub{layer}{The label for the base layer}
\defSub{cell\_length}{The length (distance) of one side of a cell}
\defSub{categories} {Labels of the categories (rows) of the population component of the partition}
\defSub{min\_age}{Minimum age of the population}
\defSub{max\_age}{Maximum age of the population}
\defSub{age\_plus\_group}{Define the largest age as a plus group}
\defSub{age\_size}{Define the label of the associated age-size relationship for each category}
\defSub{initialisation\_phases}{Define the labels of the phases of the initialisation}
\defSub{initial\_year}{Define the first year of the model, immediately following initialisation}
\defSub{current\_year}{Define the current year of the model}
\defSub{time\_steps} {Define the \command{time\_step} labels (in order that they are applied) to form the annual cycle}
\par\defComLab{initialisation\_phase}{Define the processes and years of the initialisation phase with label}\par
\defSub{years} {Define the number of years to run}
\defSub{time\_steps} {Define the \command{time\_step} labels (in order that they are applied) in this initialisation phase}
\defSub{lambda} {Define the absolute proportional difference for assessing convergence between annual iterations during the initialisation}
\defSub{lambda\_years} {Define the years to test for convergence during the initialisation}
\par\defComLab{time\_step} {Define a time-step with label}\par
\defSub{processes} {Define the process labels, in the order that they are applied, for the time-step}
\defSub{growth\_proportion} {Define the amount of growth (as a proportion of a year) to add to the age of an individual to calculate its size and weight in that time-step}
\par\defComLab{process} {Define a process with label}\par
\defSub{type} {Define the type of process}
\par\textbf{\commandlabsubarg{process}{type}{none}}\par
\par\textbf{\commandlabsubarg{process}{type}{constant\_recruitment}}\par
\defSub{r0} {Define the total amount of recruitment at equilibrium abundance levels}
\defSub{categories} {Define the categories into which recruitment occurs}
\defSub{proportions} {Define the proportion of recruitment that occurs into each category}
\defSub{age} {Define the age that receives recruitment}
\defSub{layer} {Name of the layer used to determine where recruitment occurs}
\par\textbf{\commandlabsubarg{process}{type}{bh\_recruitment}}\par
\defSub{r0} {Define the total amount of recruitment at equilibrium abundance levels}
\defSub{categories} {Define the categories into which recruitment occurs}
\defSub{proportions} {Define the proportion of recruitment that occurs into each category}
\defSub{age} {Define the age that receives recruitment}
\defSub{steepness} {Define the Beverton-Holt stock recruitment relationship steepness ($h$) parameter}
\defSub{b0} {Define the \command{initialisation\_phase} label for the value of the derived quantity  to use as the value of the spawning stock biomass ($B_0$)}
\defSub{ssb} {Define the label of the \command{derived\_quantity} that defines the spawning stock biomass (SSB)}
\defSub{ssb\_offset} {Define the offset (in years) for the year of the derived quantity that is to be applied as the SSB in the stock-recruit relationship}
\defSub{ycs\_values} {YCS values}
\defSub{standardise\_ycs\_years} {Years for which the year class strength values are defined to have mean 1.0}
\defSub{layer} {Name of the layer used to determine where recruitment occurs}
\par\textbf{\commandlabsubarg{process}{type}{bh\_recruitment2}}\par
\defSub{categories} {Define the categories into which recruitment occurs}
\defSub{proportions} {Define the proportion of recruitment that occurs into each category}
\defSub{age} {Define the age that receives recruitment}
\defSub{steepness} {Define the Beverton-Holt stock recruitment relationship steepness ($h$) parameter}
\defSub{b0} {Define the \command{initialisation\_phase} label for the value of the derived quantity  to use as the value of the spawning stock biomass ($B_0$)}
\defSub{ssb} {Define the label of the \command{derived\_quantity} that defines the spawning stock biomass (SSB)}
\defSub{ssb\_offset} {Define the offset (in years) for the year of the derived quantity that is to be applied as the SSB in the stock-recruit relationship}
\defSub{values} {The amount of recruitment in each year in \texttt{years}}
\defSub{years} {Years from which recruitments are specified in \texttt{values}}
\defSub{standardise\_recruitment\_years} {Years from which recruitments are used to derive $R_0$}
\defSub{layer} {Name of the layer used to determine where recruitment occurs}
\par\textbf{\commandlabsubarg{process}{type}{local\_bh\_recruitment}}\par
\defSub{r0} {Define a multiplier of \subcommand{r0\_layer} for calculating the amount of recruitment in each cell at equilibrium abundance levels}
\defSub{categories} {Define the categories into which recruitment occurs}
\defSub{proportions} {Define the proportion of recruitment that occurs into each category}
\defSub{age} {Define the age that receives recruitment}
\defSub{steepness} {Define the Beverton-Holt stock recruitment relationship steepness ($h$) parameter}
\defSub{b0} {Define the \command{initialisation\_phase} label for the value of the derived quantity  to use as the value of the spawning stock biomass ($B_0$) in each cell}
\defSub{ssb} {Define the label of the \command{derived\_quantity\_by\_cell} that defines the spawning stock biomass (SSB) in each cell}
\defSub{ssb\_offset} {Define the offset (in years) for the year of the derived quantity by cell that is to be applied as the SSB in the stock-recruit relationship}
\defSub{ycs\_values} {YCS values}
\defSub{standardise\_ycs\_years} {Years for which the year class strength values are defined to have mean 1.0}
\defSub{layer} {Define the label of the layer that defines the distribution of recruitment (as a multiplier of $R_0$ at equilibrium abundances in each cell}
\par\textbf{\commandlabsubarg{process}{type}{ageing}}\par
\defSub{categories} {Define the categories that ageing is applied to}
\par\textbf{\commandlabsubarg{process}{type}{constant\_mortality\_rate}}\par
\defSub{m} {Define the constant mortality rate to be applied}
\defSub{categories} {Define the categories that mortality is applied to}
\defSub{selectivities} {Define the selectivities applied to each category}
\defSub{layer} {Name of the layer}
\par\textbf{\commandlabsubarg{process}{type}{age\_mortality\_rate}}\par
\defSub{m} {Define the age specific mortality rate to be applied}
\defSub{categories} {Define the categories that mortality is applied to}
\defSub{layer} {Name of the layer}
\par\textbf{\commandlabsubarg{process}{type}{constant\_exploitation\_rate}}\par
\defSub{u} {Define the constant exploitation rate to be applied}
\defSub{u\_max} {Define the maximum constant exploitation rate that could be applied}
\defSub{categories} {Define the categories that mortality is applied to}
\defSub{selectivities} {Define the selectivities applied to each category}
\defSub{layer} {Name of the layer}
\par\textbf{\commandlabsubarg{process}{type}{annual\_mortality\_rate}}\par
\defSub{years} {Define the years when the mortality rates are applied}
\defSub{m} {Define the mortality rate to be applied for each year}
\defSub{categories} {Define the categories that mortality is applied to}
\defSub{selectivities} {Define the selectivities applied to each category}
\defSub{layer} {Name of the multiplicative layer to be applied to $M$}
\par\textbf{\commandlabsubarg{process}{type}{layer\_varying\_exploitation\_rate}}\par
\defSub{u} {Define the constant exploitation rate to be applied}
\defSub{u\_max} {Define the maximum constant exploitation rate that could be applied}
\defSub{categories} {Define the categories that mortality is applied to}
\defSub{selectivities} {Define the selectivities applied to each category}
\defSub{years} {Define the years when the exploitation rates are applied}
\defSub{layers} {Names of the layers}
\par\textbf{\commandlabsubarg{process}{type}{event\_mortality}}\par
\defSub{categories} {Define the categories that the event mortality is applied to}
\defSub{years} {Define the years where the mortality even is applied}
\defSub{layers} {Define the layers that specify the event mortality (as the abundance) in each year}
\defSub{u\_max}{Define the maximum exploitation rate}
\defSub{selectivities} {Define the selectivities applied to each category}
\defSub{penalty} {Define the event mortality penalty label}
\par\textbf{\commandlabsubarg{process}{type}{biomass\_event\_mortality}}\par
\defSub{categories}{Define the categories that the event mortality is applied to}
\defSub{years}{Define the years where the mortality event is applied}
\defSub{layers}{Define the layers that specify the event mortality (as a biomass) in each year}
\defSub{u\_max} {Define the maximum exploitation rate}
\defSub{selectivities}{Define the selectivities applied to each category}
\defSub{penalty} {Define the event mortality penalty label}
\par\textbf{\commandlabsubarg{process}{type}{Holling\_mortality\_rate}}\par
\defSub{is\_abundance}{Is the mortality applied as a biomass or as abundance}
\defSub{a}{Define the $a$ parameter of the Holling function}
\defSub{b}{Define the $b$ parameter of the Holling function}
\defSub{x}{Define the type of Holling function or Michaelis-Menton function}
\defSub{categories}{Define the categories that the Holling mortality rate is applied to}
\defSub{selectivities}{Define the selectivities applied to each category}
\defSub{predator\_categories}{Define the categories of the predator}
\defSub{predator\_selectivities}{Define the selectivities applied to each predator category}
\defSub{u\_max} {Define the maximum exploitation rate}
\defSub{penalty} {Define the event mortality penalty label}
\par\textbf{\commandlabsubarg{process}{type}{Prey-suitability\_predation}}\par
\defSub{is\_abundance}{Is the mortality applied as a biomass or as abundance}
\defSub{consumption\_rate}{Define the total predator consumption rate}
\defSub{consumption\_rate\_layer}{Define the label of the layer that defines the predator consumption rate in each cell}
\defSub{prey\_categories}{Define the prey categories that the predation mortality is applied to}
\defSub{prey\_selectivities}{Define the selectivities applied to each prey category}
\defSub{electivities}{Define the electivities applied to prey groups $1$ \ldots $n$}
\defSub{predator\_categories}{Define the categories of the predator}
\defSub{predator\_selectivities}{Define the selectivities applied to each predator category}
\defSub{u\_max} {Define the maximum exploitation rate}
\defSub{penalty} {Define the process penalty label}
\par\textbf{\commandlabsubarg{process}{type}{category\_state\_by\_age}}\par
\defSub{category} {Define the category that is the object of the process}
\defSub{layer} {Name of the categorical layer used to group the spatial cells for the process}
\defSub{min\_age} {Define the minimum age for the process }
\defSub{max\_age} {Define the maximum age for the process}
\defSub{data [label]}{Define the following data as the number of individuals to move in each age class}
\par\textbf{\commandlabsubarg{process}{type}{category\_transition}}\par
\defSub{from} {Define the categories that are the source of the transition process}
\defSub{selectivities} {Define the selectivities applied to the source categories}
\defSub{to} {Define the categories that are the sink of the transition process}
\defSub{years} {Define the years where the category transition is applied}
\defSub{layers} {Define the layers that specify the transitions (as N for each cell) in each year}
\defSub{u\_max} {Define the maximum proportion of individuals that can be moved}
\defSub{penalty} {Define the penalty to encourage models parameter values away from those which result in not enough individuals to move}
\par\textbf{\commandlabsubarg{process}{type}{category\_transition\_rate}}\par
\defSub{from} {Define the category that is the source of the transition process}
\defSub{selectivities} {Define the selectivities applied to the source categories}
\defSub{to} {Define the category that is the sink of the transition process}
\defSub{proportions} {Define the proportion of individuals to move}
\defSub{layer} {Name of the layer}
\par\textbf{\commandlabsubarg{process}{type}{category\_transition\_by\_age}}\par
\defSub{from} {Define the categories that are the source of the transition process}
\defSub{to} {Define the categories that are the sink of the transition process}
\defSub{year} {Define the year when the category transition is applied}
\defSub{layer} {Name of the categorical layer used to group the spatial cells for the process}
\defSub{min\_age} {Define the minimum age for the process }
\defSub{max\_age} {Define the maximum age for the process}
\defSub{data [label]}{Define the following data as the number of individuals to move in each age class}
\defSub{u\_max} {Define the maximum proportion of individuals that can be moved}
\defSub{penalty} {Define the penalty to encourage models parameter values away from those which result in not enough individuals to move}
\par\textbf{\commandlabsubarg{process}{type}{migration}}\par
\defSub{categories} {Define the categories that the migration movement event is applied to}
\defSub{selectivities} {Define the selectivities applied to each category}
\defSub{proportion} {Define the constant multiplier for the proportion of individuals that migrate}
\defSub{source\_layer} {Define the label of a layer that defines the source cells of the migration movement event}
\defSub{sink\_layer} {Define the label of a layer that defines the sink cells of the migration movement event}
\par\textbf{\commandlabsubarg{process}{type}{adjacent\_cell}}\par
\defSub{categories} {Define the categories that the adjacent cell movement event is applied to}
\defSub{selectivities} {Define the selectivities applied to each category}
\defSub{layer} {Define the label of a gradient layer that defines the relative strength of movement to adjacent cells}
\defSub{proportion} {Define the constant multiplier for the proportion that moves from each cell to the neighbouring cell}
\par\textbf{\commandlabsubarg{process}{type}{preference}}\par
\defSub{categories} {Define the categories that the preference function movement is applied to}
\defSub{proportion} {Define the constant multiplier for the proportion that the preference function movement is applied to}
\defSub{preference\_functions} {Define the labels of the individual  preference functions that make up the total preference function}
\par\textbf{\commandlabsubarg{process}{type}{preference\_threaded}}\par
\defSub{categories} {Define the categories that the preference function movement is applied to}
\defSub{proportion} {Define the constant multiplier for the proportion that the preference function movement is applied to}
\defSub{preference\_functions} {Define the labels of the individual  preference functions that make up the total preference function}
\par\defComLab{preference\_function} {Define a preference function with label}\par
\defSub{type} {Define the type of preference function}
\par\textbf{\commandlabsubarg{preference\_function}{type}{constant}}\par
\defSub{layer} {Defines the layer which supplies the preference function independent variable}
\defSub{alpha} {Defines the multiplicative constant $\alpha$}
\par\textbf{\commandlabsubarg{preference\_function}{type}{normal}}\par
\defSub{layer} {Defines the layer which supplies the preference function independent variable}
\defSub{alpha} {Defines the multiplicative constant $\alpha$}
\defSub{mu} {Defines the $\mu$ parameter of the normal preference function}
\defSub{sigma} {Defines the $\sigma$ parameter of the normal preference function}
\par\textbf{\commandlabsubarg{preference\_function}{type}{double\_normal}}\par
\defSub{layer} {Defines the layer which supplies the preference function independent variable}
\defSub{alpha} {Defines the multiplicative constant $\alpha$}
\defSub{mu} {Defines the $\mu$ parameter of the double-normal preference function}
\defSub{sigma\_l} {Defines the $\sigma_L$ parameter of the double-normal preference function}
\defSub{sigma\_r} {Defines the $\sigma_R$ parameter of the double-normal preference function}
\par\textbf{\commandlabsubarg{preference\_function}{type}{logistic}}\par
\defSub{layer} {Defines the layer which supplies the preference function independent variable}
\defSub{alpha} {Defines the multiplicative constant $\alpha$}
\defSub{a50} {Defines the $a_{50}$ parameter of the logistic preference function}
\defSub{ato95} {Defines the $a_{to95}$ parameter of the logistic preference function}
\par\textbf{\commandlabsubarg{preference\_function}{type}{inverse\_logistic}}\par
\defSub{layer} {Defines the layer which supplies the preference function independent variable}
\defSub{alpha} {Defines the multiplicative constant $\alpha$}
\defSub{a50} {Defines the $a_{50}$ parameter of the inverse-logistic preference function}
\defSub{ato95} {Defines the $a_{to95}$ parameter of the inverse-logistic preference function}
\par\textbf{\commandlabsubarg{preference\_function}{type}{exponential}}\par
\defSub{layer} {Defines the layer which supplies the preference function independent variable}
\defSub{alpha} {Defines the multiplicative constant $\alpha$}
\defSub{lambda} {Defines the $\lambda$ parameter of the exponential preference function}
\par\textbf{\commandlabsubarg{preference\_function}{type}{threshold}}\par
\defSub{layer} {Defines the layer which supplies the preference function independent variable}
\defSub{alpha} {Defines the multiplicative constant $\alpha$}
\defSub{n} {Defines the $N$ parameter of the threshold preference function}
\defSub{lambda} {Defines the $\lambda$ parameter of the threshold preference function}
\par\textbf{\commandlabsubarg{preference\_function}{type}{knife\_edge}}\par
\defSub{layer} {Defines the layer which supplies the preference function independent variable}
\defSub{alpha} {Defines the multiplicative constant $\alpha$}
\defSub{c} {Defines the $c$ parameter of the knife-edge preference function}
\par\textbf{\commandlabsubarg{preference\_function}{type}{categorical}}\par
\defSub{layer} {Defines the layer which supplies the preference function independent variable}
\defSub{alpha} {Defines the multiplicative constant $\alpha$}
\defSub{category\_labels} {Defines the unique labels of \argument{layer} in order of their coefficients}
\defSub{category\_values} {Defines the coefficients for each unique label of \argument{layer} in order of their labels}
\par\textbf{\commandlabsubarg{preference\_function}{type}{monotonic\_categorical}}\par
\defSub{layer} {Defines the layer which supplies the preference function independent variable}
\defSub{alpha} {Defines the multiplicative constant $\alpha$}
\defSub{category\_labels} {Defines the unique labels of \argument{layer} in order of their coefficients}
\defSub{category\_values} {Defines the coefficients for each unique label of \argument{layer} in order of their labels}
\par\textbf{\commandlabsubarg{preference\_function}{type}{gaussian\_copula}}\par
\defSub{rho} {Defines the dependence parameter ($\rho$) for the copula}
\defSub{layers} {Defines the two layers which supplies the preference function independent variables}
\defSub{pdf} {Defines the two PDFs for the copula}
\par\textbf{\commandlabsubarg{preference\_function}{type}{gumbel\_copula}}\par
\defSub{rho} {Defines the dependence parameter ($\rho$) for the copula}
\defSub{layers} {Defines the two layers which supplies the preference function independent variables}
\defSub{pdf} {Defines the two PDFs for the copula}
\par\textbf{\commandlabsubarg{preference\_function}{type}{frank\_copula}}\par
\defSub{rho} {Defines the dependence parameter ($\rho$) for the copula}
\defSub{layers} {Defines the two layers which supplies the preference function independent variables}
\defSub{pdf} {Defines the two PDF for the copula}
\par\textbf{\commandlabsubarg{preference\_function}{type}{independence\_copula}}\par
\defSub{layers} {Defines the two layers which supplies the preference function independent variables}
\defSub{pdf} {Defines the two PDFs for the copula}
\par\defComLab{pdf} {Define a PDF for use as a copula preference function with label}\par
\defSub{type} {Define the type of PDF}
\par\textbf{\commandlabsubarg{pdf}{type}{normal}}\par
\defSub{mu} {Define the mean of the PDF}
\defSub{sigma} {Define the variance of the PDF}
\par\textbf{\commandlabsubarg{pdf}{type}{lognormal}}\par
\defSub{mu} {Define the mean of the PDF}
\defSub{sigma} {Define the variance of the PDF}
\par\textbf{\commandlabsubarg{pdf}{type}{exponential}}\par
\defSub{lambda} {Define the mean of the PDF}
\par\textbf{\commandlabsubarg{pdf}{type}{uniform}}\par
\defSub{a} {Define the minimum of the PDF}
\defSub{b} {Define the maximum of the PDF}
\par\defComLab{layer} {Define a layer function with label}\par
\defSub{type} {Define the type of layer}
\par\textbf{\commandlabsubarg{layer}{type}{numeric}}\par
\defSub{data} {Define the values of the layer}
\defSub{rescale} {If defined, then the values of the layer are rescaled to sum to this value}
\par\textbf{\commandlabsubarg{layer}{type}{categorical}}\par
\defSub{data} {Define the values of the layer}
\par\textbf{\commandlabsubarg{layer}{type}{distance}}\par
\par\textbf{\commandlabsubarg{layer}{type}{haversine}}\par
\defSub{latitude} {Define the layer that specifies the latitudes for each cell}
\defSub{longitude} {Define the layer that specifies the longitudes for each cell}
\par\textbf{\commandlabsubarg{layer}{type}{dijkstra}}\par
\par\textbf{\commandlabsubarg{layer}{type}{haversine\_dijkstra}}\par
\defSub{latitude} {Define the layer that specifies the latitudes for each cell}
\defSub{longitude} {Define the layer that specifies the longitudes for each cell}
\par\textbf{\commandlabsubarg{layer}{type}{abundance}}\par
\defSub{categories} {Define the categories are used to calculate the abundance}
\defSub{selectivities} {Define the selectivities applied to each category}
\par\textbf{\commandlabsubarg{layer}{type}{biomass}}\par
\defSub{categories} {Define the categories are used to calculate the biomass}
\defSub{selectivities} {Define the selectivities applied to each category}
\par\textbf{\commandlabsubarg{layer}{type}{abundance\_density}}\par
\defSub{categories} {Define the categories are used to calculate the abundance}
\defSub{selectivities} {Define the selectivities applied to each category}
\par\textbf{\commandlabsubarg{layer}{type}{biomass\_density}}\par
\defSub{categories} {Define the categories are used to calculate the biomass}
\defSub{selectivities} {Define the selectivities applied to each category}
\par\textbf{\commandlabsubarg{layer}{type}{numeric\_meta}}\par
\defSub{default\_layer} {Define the default layer to use in years or initialisation phases where it is not otherwise defined}
\defSub{years} {Define the years that have a non-default layer}
\defSub{layers} {Define the layers for each of the years}
\defSub{initialisation\_phases} {Define the initialisation phases that have a non-default layer}
\defSub{initialisation\_layers} {Define the layers for each of the initialisation phases}
\par\textbf{\commandlabsubarg{layer}{type}{categorical\_meta}}\par
\defSub{default\_layer} {Define the default layer to use in years or initialisation phases where it is not otherwise defined}
\defSub{years} {Define the years that have a non-default layer}
\defSub{layers} {Define the layers for each of the years}
\defSub{initialisation\_phases} {Define the initialisation phases that have a non-default layer}
\defSub{initialisation\_layers} {Define the layers for each of the initialisation phases}
\par\textbf{\commandlabsubarg{layer}{type}{derived\_quantity}}\par
\defSub{derived\_quantity} {Define the label of the \command{derived\_quantity} that is used as the source for the layer}
\defSub{year\_offset} {Define the offset (in years) for the year of the derived quantity that is to be applied}
\par\textbf{\commandlabsubarg{layer}{type}{derived\_quantity\_by\_cell}}\par
\defSub{derived\_quantity\_by\_cell} {Define the label of the \command{derived\_quantity\_by\_cell} that is used as the source for the layer}
\defSub{year\_offset} {Define the offset (in years) for the year of the derived quantity by cell that is to be applied}
\par\defComLab{derived\_quantity\_by\_cell} {Define a derived quantity by cell with label}\par
\defSub{type} {Define the type of derived quantity by cell}
\par\textbf{\commandlabsubarg{derived\_quantity\_by\_cell}{type}{abundance}}\par
\defSub{categories} {Define the categories are used to calculate the derived quantity by cell}
\defSub{selectivities} {Define the selectivities}
\defSub{initialisation\_time\_steps} {Define the time-steps during the initialisation phases at the end of which the derived quantity by cell is calculated}
\defSub{time\_step} {Define the time-step at the end of which the derived quantity by cell is calculated}
\defSub{layer} {Define the layer to be used in the calculations}
\par\textbf{\commandlabsubarg{derived\_quantity\_by\_cell}{type}{biomass}}\par
\defSub{categories} {Define the categories are used to calculate the derived quantity by cell}
\defSub{selectivities} {Define the selectivities}
\defSub{initialisation\_time\_steps} {Define the time-steps during the initialisation phases at the end of which the derived quantity by cell is calculated}
\defSub{time\_step} {Define the time-step at the end of which the derived quantity by cell is calculated}
\defSub{layer} {Define the layer to be used in the calculations}
\par\defComLab{derived\_quantity} {Define a derived quantity with label}\par
\defSub{type} {Define the type of derived quantity}
\par\textbf{\commandlabsubarg{derived\_quantity}{type}{abundance}}\par
\defSub{categories} {Define the categories are used to calculate the derived quantity}
\defSub{selectivities} {Define the selectivities}
\defSub{initialisation\_time\_steps} {Define the time-steps during the initialisation phases at the end of which the derived quantity is calculated}
\defSub{time\_step} {Define the time-step at the end of which the derived quantity is calculated}
\defSub{layer} {Define the layer to be used in the calculations}
\par\textbf{\commandlabsubarg{derived\_quantity}{type}{biomass}}\par
\defSub{categories} {Define the categories are used to calculate the derived quantity}
\defSub{selectivities} {Define the selectivities}
\defSub{initialisation\_time\_steps} {Define the time-steps during the initialisation phases at the end of which the derived quantity is calculated}
\defSub{time\_step} {Define the time-step at the end of which the derived quantity is calculated}
\defSub{layer} {Define the layer to be used in the calculations}
\par\defComLab{age\_size} {Define an age-size relationship with label}\par
\defSub{type} {Define the type of size-at-age relationship}
\par\textbf{\commandlabsubarg{age\_size}{type}{none}}\par
\par\textbf{\commandlabsubarg{age\_size}{type}{von\_bertalanffy}}\par
\defSub{linf} {Define the $L_\infty$ parameter of the von Bertalanffy relationship}
\defSub{k} {Define the $k$ parameter of the von Bertalanffy relationship}
\defSub{t0} {Define the $t_0$ parameter of the von Bertalanffy relationship}
\defSub{distribution} {Define the distribution of sizes-at-age around the mean}
\defSub{by\_length} {Specifies if the linear interpolation of c.v.s is a linear function of mean size or of age}
\defSub{cv} {Define the c.v. of the distribution of sizes-at-age around the mean}
\defSub{size\_weight} {Define the label of the associated size-weight relationship}
\par\textbf{\commandlabsubarg{age\_size}{type}{schnute}}\par
\defSub{y1} {Define the $y_1$ parameter of the Schnute relationship}
\defSub{y2} {Define the $y_2$ parameter of the Schnute relationship}
\defSub{tau1} {Define the $\tau_1$ parameter of the Schnute relationship}
\defSub{tau2} {Define the $\tau_2$ parameter of the Schnute relationship}
\defSub{a} {Define the $a$ parameter of the Schnute relationship}
\defSub{b} {Define the $b$ parameter of the Schnute relationship}
\defSub{distribution} {Define the distribution of sizes-at-age around the mean}
\defSub{by\_length} {Specifies if the linear interpolation of c.v.s is a linear function of mean size or of age}
\defSub{cv} {Define the c.v. of the distribution of sizes-at-age around the mean}
\defSub{size\_weight} {Define the label of the associated size-weight relationship}
\par\textbf{\commandlabsubarg{age\_size}{type}{von\_bertalanffy\_k\_at\_age}}\par
\defSub{linf} {Define the $L_\infty$ parameter of the von Bertalanffy k-at-age relationship}
\defSub{Lmin} {Define the $L_min$ parameter of the von Bertalanffy k-at-age relationship}
\defSub{k} {Define the $k$ parameter of the von Bertalanffy with k-at-age relationship}
\defSub{t0} {Define the $t_0$ parameter of the von Bertalanffy with k-at-age relationship}
\defSub{distribution} {Define the distribution of sizes-at-age around the mean}
\defSub{by\_length} {Specifies if the linear interpolation of c.v.s is a linear function of mean size or of age}
\defSub{cv} {Define the c.v. of the distribution of sizes-at-age around the mean}
\defSub{size\_weight} {Define the label of the associated size-weight relationship}
\par\textbf{\commandlabsubarg{age\_size}{type}{mean\_size\_at\_age}}\par
\defSub{sizes} {Define the mean size at age for each age class in the partition}
\defSub{distribution} {Define the distribution of sizes-at-age around the mean}
\defSub{by\_length} {Specifies if the linear interpolation of c.v.s is a linear function of mean size or of age}
\defSub{cv} {Define the c.v. of the distribution of sizes-at-age around the mean}
\defSub{size\_weight} {Define the label of the associated size-weight relationship}
\par\defComLab{size\_weight} {Define a size-weight relationship with label}\par
\defSub{type} {Define the type of relationship}
\par\textbf{\commandlabsubarg{size\_weight}{type}{none}}\par
\par\textbf{\commandlabsubarg{size\_weight}{type}{basic}}\par
\defSub{a} {Define the $a$ parameter of the basic size-weight relationship}
\defSub{b} {Define the $b$ parameter of the basic size-weight relationship}
\par\defComLab{selectivity} {Define a selectivity function with label}\par
\defSub{type} {Define the type of selectivity function}
\par\textbf{\commandlabsubarg{selectivity}{type}{constant}}\par
\defSub{c} {Defines the $C$ parameter of the selectivity function}
\par\textbf{\commandlabsubarg{selectivity}{type}{knife\_edge}}\par
\defSub{e} {Defines the $E$ parameter of the selectivity function}
\par\textbf{\commandlabsubarg{selectivity}{type}{all\_values}}\par
\defSub{v} {Defines the $V$ parameters (one for each age class) of the selectivity function}
\par\textbf{\commandlabsubarg{selectivity}{type}{all\_values\_bounded}}\par
\defSub{l} {Defines the $L$ parameter of the selectivity function}
\defSub{h} {Defines the $H$ parameter of the selectivity function}
\defSub{v} {Defines the $V$ parameters (one for each age class from $L$ to $H$) of the selectivity function}
\par\textbf{\commandlabsubarg{selectivity}{type}{increasing}}\par
\defSub{alpha} {Defines the $\alpha$ parameter of the selectivity function}
\defSub{l} {Defines the $L$ parameter of the selectivity function}
\defSub{h} {Defines the $H$ parameter of the selectivity function}
\defSub{v} {Defines the $V$ parameters (one for each age class from $L$ to $H$) of the selectivity function}
\par\textbf{\commandlabsubarg{selectivity}{type}{logistic}}\par
\defSub{alpha} {Defines the $\alpha$ parameter of the selectivity function}
\defSub{a50} {Defines the $a_{50}$ parameter of the selectivity function}
\defSub{ato95} {Defines the $a_{to95}$ parameter of the selectivity function}
\par\textbf{\commandlabsubarg{selectivity}{type}{inverse\_logistic}}\par
\defSub{alpha} {Defines the $\alpha$ parameter of the selectivity function}
\defSub{a50} {Defines the $a_{50}$ parameter of the selectivity function}
\defSub{ato95} {Defines the $a_{to95}$ parameter of the selectivity function}
\par\textbf{\commandlabsubarg{selectivity}{type}{logistic\_producing}}\par
\defSub{alpha} {Defines the $\alpha$ parameter of the selectivity function}
\defSub{l} {Defines the $L$ parameter of the selectivity function}
\defSub{h} {Defines the $H$ parameter of the selectivity function}
\defSub{a50} {Defines the $a_{50}$ parameter of the selectivity function}
\defSub{ato95} {Defines the $a_{to95}$ parameter of the selectivity function}
\par\textbf{\commandlabsubarg{selectivity}{type}{double\_normal}}\par
\defSub{alpha} {Defines the $\alpha$ parameter of the selectivity function}
\defSub{mu} {Defines the $\mu$ parameter of the selectivity function}
\defSub{sigma\_l} {Defines the $\sigma_L$ parameter of the selectivity function}
\defSub{sigma\_r} {Defines the $\sigma_R$ parameter of the selectivity function}
\par\textbf{\commandlabsubarg{selectivity}{type}{double\_exponential}}\par
\defSub{alpha} {Defines the $\alpha$ parameter of the selectivity function}
\defSub{x1} {Defines the $x_1$ parameter of the selectivity function}
\defSub{x2} {Defines the $x_2$ parameter of the selectivity function}
\defSub{x0} {Defines the $x_0$ parameter of the selectivity function}
\defSub{y0} {Defines the $y_0$ parameter of the selectivity function}
\defSub{y1} {Defines the $y_1$ parameter of the selectivity function}
\defSub{y2} {Defines the $y_2$ parameter of the selectivity function}
\par\textbf{\commandlabsubarg{selectivity}{type}{spline}}\par
\defSub{alpha} {Defines the $\alpha$ parameter of the selectivity function}
\defSub{knots} {Defines the locations of the knots for the cubic spline function}
\defSub{values} {Defines the values at the knots for the cubic spline function}
\defSub{method} {The method for constraining the end values of the spline}
\subsection{Estimation command and subcommand syntax}\par
\defCom{estimation}\par\par
\defSub{minimiser} {The label of the minimiser to use, if doing a point estimate}
\defSub{mcmc} {The label of the MCMC to use, if doing an MCMC}
\defSub{profile} {The labels of the profiles to use, if doing a profile}
\par\defComLab{minimiser}{Define the an minimiser estimator with label}\par\par
\defSub{type} {Define the type of minimiser}
\par\textbf{\commandlabsubarg{minimiser}{type}{numerical\_differences}}\par
\defSub{iterations} {Define the maximum number of iterations for the minimiser}
\defSub{evaluations} {Define the maximum number of evaluations for the minimiser}
\defSub{stepsize} {Define the stepsize for the minimiser}
\defSub{tolerance} {Define the convergence criteria (tolerance) for the minimiser}
\defSub{covariance} {Specify if \SPM\ should attempt to calculate the covariance matrix, if estimating}
\defSub{transform\_method} {Specify the method for transforming parameters to ensure values remain within bounds}
\par\textbf{\commandlabsubarg{minimiser}{type}{de\_solver}}\par
\defSub{population\_size} {Define the minimisers number of populations to generate}
\defSub{crossover\_probability} {Define the minimisers crossover probability }
\defSub{difference\_scale} {Define the scale of the difference of the parent candidates for the minimiser}
\defSub{max\_generations} {Define the maximum generations for the minimiser convergence}
\defSub{tolerance} {Define the convergence criteria (tolerance) for the minimiser}
\defSub{covariance} {Specify if \SPM\ should attempt to calculate the covariance matrix, if estimating}
\par\defComLab{mcmc}{Define the MCMC estimation arguments}\par\par
\defSub{type} {Define the method of MCMC}
\par\textbf{\commandsubarg{mcmc}{type}{metropolis\_hastings}}\par
\defSub{start} {Covariance multiplier for the starting point of the Markov chain}
\defSub{length} {Length of the Markov chain}
\defSub{keep} {Spacing between recorded values in the chain}
\defSub{max\_correlation} {Maximum absolute correlation in the covariance matrix of the proposal distribution}
\defSub{covariance\_adjustment\_method} {Method for adjusting small variances in the covariance proposal matrix}
\defSub{correlation\_adjustment\_diff} {Minimum non-zero variance times the range of the bounds in the covariance matrix of the proposal distribution}
\defSub{stepsize} {Initial stepsize (as a multiplier of the approximate covariance matrix)}
\defSub{proposal\_distribution} {The shape of the proposal distribution (either \textit{t} or normal)}
\defSub{df} {Degrees of freedom of the multivariate t proposal distribution}
\defSub{adapt\_stepsize\_at} {Iterations in the chain to check and resize the MCMC stepsize}
\defSub{acceptance\_ratio} {Target acceptance ratio for modifying the stepsize during adaptive updates in the MCMC chain}
\par\defComLab{profile}{Define the profile parameters and arguments}\par\par
\defSub{parameter} {Name of the parameter to be profiled}
\defSub{steps} {Number of steps (values) at which to profile the parameter}
\defSub{lower\_bound} {lower bound on parameter}
\defSub{upper\_bound} {Upper bound on parameter}
\par\defComArg{estimate}{parameter\_name}{Estimate an estimable parameter}\par\par
\defSub{same}{Names of the other parameters which are constrained to have the same value}
\defSub{lower\_bound}{Lower bounds on this parameter}
\defSub{upper\_bound}{Upper bound on this parameter}
\defSub{mcmc\_fixed}{Should this parameter be fixed during MCMC?}
\defSub{type}{Defines the type of prior for this parameter}
\par\textbf{\commandlabsubarg{estimate}{type}{uniform}}\par
\par\textbf{\commandlabsubarg{estimate}{type}{uniform\_log}}\par
\par\textbf{\commandlabsubarg{estimate}{type}{normal}}\par
\defSub{mu}{Defines the mean $\mu$ of the normal prior}
\defSub{cv}{Defines the c.v. $c$ of the normal prior}
\par\textbf{\commandlabsubarg{estimate}{type}{normal\_by\_stdev}}\par
\defSub{mu}{Defines the mean $\mu$ of the normal by standard deviation prior}
\defSub{sigma}{Defines the standard deviation $\sigma$ of the normal by standard deviation prior}
\par\textbf{\commandlabsubarg{estimate}{type}{lognormal}}\par
\defSub{mu}{Defines the mean $\mu$ of the lognormal prior}
\defSub{cv}{Defines the c.v. $c$ of the lognormal prior}
\par\textbf{\commandlabsubarg{estimate}{type}{beta}}\par
\defSub{a}{The lower value of the range parameter $A$ of the Beta prior}
\defSub{b}{The upper value of the range parameter $B$ of the Beta prior}
\defSub{mu}{Defines the mean $\mu$ of the Beta prior}
\defSub{sigma}{Defines the standard deviation $\sigma$ of the Beta prior}
\par\defComLab{catchability}{Define a catchability constant with \argument{label}}\par\par
\defSub{q} {Value of the q parameter}
\par\defComLab{penalty}{Define a penalty with \argument{label}}\par\par
\defSub{log\_scale} {Defines if the penalty in calculated in log space}
\defSub{multiplier} {Penalty multiplier}
\subsection{Observation command and subcommand syntax}\par
\defComLab{observation}{Define an observation}\par\par
\defSub{type} {Define the type of observation}
\par\textbf{\commandlabsubarg{observation}{type}{proportions\_at\_age}}\par
\defSub{year} {Define the year that the observation applies to}
\defSub{time\_step} {Define the time-step that the observation applies to}
\defSub{proportion\_method} {Define the method for interpolating the time-step for calculating the expected value of the observation}
\defSub{proportion\_time\_step} {Define the interpolated proportion through the time-step for calculating the expected value of the observation}
\defSub{categories} {Define the categories}
\defSub{selectivities} {Define the selectivities applied to each category}
\defSub{min\_age} {Define the minimum age for the observation}
\defSub{max\_age} {Define the maximum age for the observation}
\defSub{age\_plus\_group} {Define if the maximum age for the observation is a plus group}
\defSub{ageing\_error} {Define the label of the ageing-error matrix to be applied (if any)}
\defSub{layer} {Name of the categorical layer used to group the spatial cells for the observation}
\defSub{obs [label]}{Define the following data as observations for the categorical layer with value \texttt{[label]}}
\defSub{tolerance}{Define the tolerance on the sum-to-one error check in \SPM}
\defSub{error\_value [label]}{Define the following data as error values (e.g., $N$ for multinomial likelihoods, c.v. for lognormal likelihoods, etc.) for the categorical layer with value \texttt{[label]}}
\defSub{likelihood}{Define the likelihood for the observation}
\defSub{delta}{Define the delta robustifying constant for the likelihood}
\defSub{process\_error}{Define the process error term}
\defSub{likelihood\_multiplier}{Define the scaler applied to the likelihood}
\defSub{simulation\_likelihood}{Define the likelihood when doing simulations, if the observations is a pseudo-observation}
\par\textbf{\commandlabsubarg{observation}{type}{proportions\_by\_category}}\par
\defSub{year} {Define the year that the observation applies to}
\defSub{time\_step} {Define the time-step that the observation applies to}
\defSub{proportion\_method} {Define the method for interpolating the time-step for calculating the expected value of the observation}
\defSub{proportion\_time\_step} {Define the interpolated proportion through the time-step for calculating the expected value of the observation}
\defSub{categories} {Define the categories that make up the numerator of the observation}
\defSub{categories2} {Define the categories that, in combination with the numerator categories, make up the denominator}
\defSub{selectivities} {Define the selectivities applied to each category}
\defSub{selectivities2} {Define the selectivities applied to each category}
\defSub{min\_age} {Define the minimum age for the observation}
\defSub{max\_age} {Define the maximum age for the observation}
\defSub{age\_plus\_group} {Define if the maximum age for the observation is a plus group}
\defSub{ageing\_error} {Define the label of the ageing-error matrix to be applied}
\defSub{layer} {Name of the categorical layer used to group the spacial cells for the observation}
\defSub{obs [label]}{Define the following data as observations for the categorical layer with value \argument{[label]}}
\defSub{error\_value [label]}{Define the following data as error values (e.g., $N$ for multinomial likelihoods, c.v. for lognormal likelihoods, etc.) for the categorical layer with value \texttt{[label]}}
\defSub{detection\_probability} {Define the detection probability for modifying the denominator when calculating the expected value of the observation}
\defSub{likelihood}{Define the likelihood for the observation}
\defSub{delta}{Define the delta robustifying constant for the likelihood}
\defSub{process\_error}{Define the process error term}
\defSub{likelihood\_multiplier}{Define the scaler applied to the likelihood}
\defSub{simulation\_likelihood}{Define the likelihood when doing simulations, if the observations is a pseudo-observation}
\par\textbf{\commandlabsubarg{observation}{type}{proportions\_at\_length}}\par
\defSub{year} {Define the year that the observation applies to}
\defSub{time\_step} {Define the time-step that the observation applies to}
\defSub{proportion\_method} {Define the method for interpolating the time-step for calculating the expected value of the observation}
\defSub{proportion\_time\_step} {Define the interpolated proportion through the time-step for calculating the expected value of the observation}
\defSub{categories} {Define the categories}
\defSub{selectivities} {Define the selectivities applied to each category}
\defSub{length\_bins} {Define the length bins for the observation}
\defSub{layer} {Name of the categorical layer used to group the spatial cells for the observation}
\defSub{obs [label]}{Define the following data as observations for the categorical layer with value \texttt{[label]}}
\defSub{tolerance}{Define the tolerance on the sum-to-one error check in \SPM}
\defSub{error\_value [label]}{Define the following data as error values (e.g., $N$ for multinomial likelihoods, c.v. for lognormal likelihoods, etc.) for the categorical layer with value \texttt{[label]}}
\defSub{likelihood}{Define the likelihood for the observation}
\defSub{delta}{Define the delta robustifying constant for the likelihood}
\defSub{process\_error}{Define the process error term}
\defSub{likelihood\_multiplier}{Define the scaler applied to the likelihood}
\defSub{simulation\_likelihood}{Define the likelihood when doing simulations, if the observations is a pseudo-observation}
\par\textbf{\commandlabsubarg{observation}{type}{proportions\_by\_category\_at\_length}}\par
\defSub{year} {Define the year that the observation applies to}
\defSub{time\_step} {Define the time-step that the observation applies to}
\defSub{proportion\_method} {Define the method for interpolating the time-step for calculating the expected value of the observation}
\defSub{proportion\_time\_step} {Define the interpolated proportion through the time-step for calculating the expected value of the observation}
\defSub{categories} {Define the categories that make up the numerator of the observation}
\defSub{categories2} {Define the categories that, in combination with the numerator categories, make up the denominator}
\defSub{selectivities} {Define the selectivities applied to each category}
\defSub{selectivities2} {Define the selectivities applied to each category}
\defSub{length\_bins} {Define the length bins for the observation}
\defSub{layer} {Name of the categorical layer used to group the spacial cells for the observation}
\defSub{obs [label]}{Define the following data as observations for the categorical layer with value \argument{[label]}}
\defSub{error\_value [label]}{Define the following data as error values (e.g., $N$ for multinomial likelihoods, c.v. for lognormal likelihoods, etc.) for the categorical layer with value \texttt{[label]}}
\defSub{detection\_probability} {Define the detection probability for modifying the denominator when calculating the expected value of the observation}
\defSub{likelihood}{Define the likelihood for the observation}
\defSub{delta}{Define the delta robustifying constant for the likelihood}
\defSub{process\_error}{Define the process error term}
\defSub{likelihood\_multiplier}{Define the scaler applied to the likelihood}
\defSub{simulation\_likelihood}{Define the likelihood when doing simulations, if the observations is a pseudo-observation}
\par\textbf{\commandlabsubarg{observation}{type}{abundance}}\par
\defSub{year} {Define the year that the observation applies to}
\defSub{time\_step} {Define the time-step that the observation applies to}
\defSub{proportion\_method} {Define the method for interpolating the time-step for calculating the expected value of the observation}
\defSub{proportion\_time\_step} {Define the interpolated proportion through the time-step for calculating the expected value of the observation}
\defSub{catchability} {Define the catchability constant label for the observation}
\defSub{categories} {Define the categories for which the observations occur}
\defSub{selectivities} {Define the selectivities applied to each category}
\defSub{layer} {Name of the categorical layer used to group the spacial cells for the observation}
\defSub{obs [label]}{Define the following data as observations for the categorical layer with value \argument{[label]}}
\defSub{error\_value [label]}{Define the following data as error values (e.g., $N$ for multinomial likelihoods, c.v. for lognormal likelihoods, etc.) for the categorical layer with value \texttt{[label]}}
\defSub{likelihood}{Define the likelihood for the observation}
\defSub{delta}{Define the delta robustifying constant for the likelihood}
\defSub{process\_error}{Define the process error term}
\defSub{likelihood\_multiplier}{Define the scaler applied to the likelihood}
\defSub{simulation\_likelihood}{Define the likelihood when doing simulations, if the observations is a pseudo-observation}
\par\textbf{\commandlabsubarg{observation}{type}{biomass}}\par
\defSub{year} {Define the year that the observation applies to}
\defSub{time\_step} {Define the time-step that the observation applies to}
\defSub{proportion\_method} {Define the method for interpolating the time-step for calculating the expected value of the observation}
\defSub{proportion\_time\_step} {Define the interpolated proportion through the time-step for calculating the expected value of the observation}
\defSub{catchability} {Define the catchability constant label for the observation}
\defSub{categories} {Define the categories for which the observations occur}
\defSub{selectivities} {Define the selectivities applied to each category}
\defSub{layer} {Name of the categorical layer used to group the spacial cells for the observation}
\defSub{obs [label]}{Define the following data as observations for the categorical layer with value \argument{[label]}}
\defSub{error\_value [label]}{Define the following data as error values (e.g., $N$ for multinomial likelihoods, c.v. for lognormal likelihoods, etc.) for the categorical layer with value \texttt{[label]}}
\defSub{likelihood}{Define the likelihood for the observation}
\defSub{delta}{Define the delta robustifying constant for the likelihood}
\defSub{process\_error}{Define the process error term}
\defSub{likelihood\_multiplier}{Define the scaler applied to the likelihood}
\defSub{simulation\_likelihood}{Define the likelihood when doing simulations, if the observations is a pseudo-observation}
\par\textbf{\commandlabsubarg{observation}{type}{presence}}\par
\defSub{year} {Define the year that the observation applies to}
\defSub{time\_step} {Define the time-step that the observation applies to}
\defSub{proportion\_method} {Define the method for interpolating the time-step for calculating the expected value of the observation}
\defSub{proportion\_time\_step} {Define the interpolated proportion through the time-step for calculating the expected value of the observation}
\defSub{catchability} {Define the catchability constant label for the observation}
\defSub{categories} {Define the categories for which the observations occur}
\defSub{selectivities} {Define the selectivities applied to each category}
\defSub{layer} {Name of the categorical layer used to group the spacial cells for the observation}
\defSub{obs [label]}{Define the following data as observations for the categorical layer with value \argument{[label]}}
\defSub{error\_value [label]}{Define the following data as error values (e.g., $N$ for binomial likelihoods) for the categorical layer with value \texttt{[label]}}
\defSub{likelihood}{Define the likelihood for the observation}
\defSub{delta}{Define the delta robustifying constant for the likelihood}
\defSub{process\_error}{Define the process error term}
\defSub{likelihood\_multiplier}{Define the scaler applied to the likelihood}
\defSub{simulation\_likelihood}{Define the likelihood when doing simulations, if the observations is a pseudo-observation}
\par\defComLab{ageing\_error}{Define ageing error with \argument{label}}\par\par
\defSub{type} {The type of ageing error}
\par\textbf{\commandlabsubarg{ageing\_error}{type}{none}}\par
\par\textbf{\commandlabsubarg{ageing\_error}{type}{normal}}\par
\defSub{cv} {Parameter of the normal ageing error model}
\defSub{k} {The $k$ parameter of the normal ageing error model}
\par\textbf{\commandlabsubarg{ageing\_error}{type}{off\_by\_one}}\par
\defSub{p1} {The $p_1$ parameter of the off-by-one ageing error model}
\defSub{p2} {The $p_2$ parameter of the off-by-one ageing error model}
\defSub{k} {The $k$ parameter of the off-by-one ageing error model}
\subsection{Report command and subcommand syntax}\par
\defComLab{report}{Define an output report}\par\par
\defSub{type} {Define the type of report}
\par\textbf{\commandlabsubarg{report}{type}{spatial\_map}}\par
\defSub{file\_name} {Define the name of the output file where the report is written}
\par\textbf{\commandlabsubarg{report}{type}{partition}}\par
\defSub{years} {Define the years that the partition report applies to}
\defSub{time\_step} {Define the time-step that the partition report applies to}
\defSub{file\_name} {Define the name of the output file where the report is written}
\defSub{overwrite} {Specify if any previous file with the same name as the output file should be overwritten or appended to}
\par\textbf{\commandlabsubarg{report}{type}{partition\_biomass}}\par
\defSub{years} {Define the years that the partition\_biomass report applies to}
\defSub{time\_step} {Define the time-step that the partition\_biomass report applies to}
\defSub{file\_name} {Define the name of the output file where the report is written}
\defSub{overwrite} {Specify if any previous file with the same name as the output file should be overwritten or appended to}
\par\textbf{\commandlabsubarg{report}{type}{initialisation}}\par
\defSub{initialisation\_phase} {Define the phase of initialisation that the partition report applies to}
\defSub{file\_name} {Define the name of the output file where the report is written}
\defSub{overwrite} {Specify if any previous file with the same name as the output file should be overwritten or appended to}
\par\textbf{\commandlabsubarg{report}{type}{process}}\par
\defSub{process} {Define the label of the process to summarise}
\defSub{file\_name} {Define the name of the output file where the report is written}
\defSub{overwrite} {Specify if any previous file with the same name as the output file should be overwritten or appended to}
\par\textbf{\commandlabsubarg{report}{type}{preference\_function}}\par
\defSub{preference\_function} {Define the label of the preference function to summarise}
\defSub{file\_name} {Define the name of the output file where the report is written}
\defSub{overwrite} {Specify if any previous file with the same name as the output file should be overwritten or appended to}
\par\textbf{\commandlabsubarg{report}{type}{derived\_quantity}}\par
\defSub{derived\_quantity} {Define the label of the derived quantity to print}
\defSub{file\_name} {Define the name of the output file where the report is written}
\defSub{overwrite} {Specify if any previous file with the same name as the output file should be overwritten or appended to}
\par\textbf{\commandlabsubarg{report}{type}{derived\_quantity\_by\_cell}}\par
\defSub{derived\_quantity\_by\_cell} {Define the label of the derived quantity by cell to print}
\defSub{initialisation} {Specify if the derived quantity by cell values for each year of the initialisation phases should be also be output}
\defSub{file\_name} {Define the name of the output file where the report is written}
\defSub{overwrite} {Specify if any previous file with the same name as the output file should be overwritten or appended to}
\par\textbf{\commandlabsubarg{report}{type}{estimate\_summary}}\par
\defSub{file\_name} {Define the name of the output file where the report is written}
\defSub{overwrite} {Specify if any previous file with the same name as the output file should be overwritten or appended to}
\par\textbf{\commandlabsubarg{report}{type}{estimate\_value}}\par
\defSub{file\_name} {Define the name of the output file where the report is written}
\defSub{header} {Specify if the output contains the standard \SPM\ style header at the start of the output}
\defSub{overwrite} {Specify if any previous file with the same name as the output file should be overwritten or appended to}
\par\textbf{\commandlabsubarg{report}{type}{objective\_function}}\par
\defSub{file\_name} {Define the name of the output file where the report is written}
\defSub{overwrite} {Specify if any previous file with the same name as the output file should be overwritten or appended to}
\par\textbf{\commandlabsubarg{report}{type}{covariance}}\par
\defSub{file\_name} {Define the name of the output file where the report is written}
\defSub{overwrite} {Specify if any previous file with the same name as the output file should be overwritten or appended to}
\par\textbf{\commandlabsubarg{report}{type}{observation}}\par
\defSub{observation} {Define the label of the observation to print}
\defSub{file\_name} {Define the name of the output file where the report is written}
\defSub{overwrite} {Specify if any previous file with the same name as the output file should be overwritten or appended to}
\par\textbf{\commandlabsubarg{report}{type}{simulated\_observation}}\par
\defSub{observation} {Define the label of the observation from which to simulate values}
\defSub{file\_name} {Define the name of the output file where the report is written}
\defSub{overwrite} {Specify if any previous file with the same name as the output file should be overwritten or appended to}
\par\textbf{\commandlabsubarg{report}{type}{ageing\_error}}\par
\defSub{ageing\_error} {Define the label of the ageing\_error misclassification matrix}
\defSub{file\_name} {Define the name of the output file where the report is written}
\defSub{overwrite} {Specify if any previous file with the same name as the output file should be overwritten or appended to}
\par\textbf{\commandlabsubarg{report}{type}{layer}}\par
\defSub{layer} {Define the label of the layer to print}
\defSub{years} {Define the years for the printing of the layer}
\defSub{time\_step} {Define the time-step for the printing of the layer}
\defSub{file\_name} {Define the name of the output file where the report is written}
\defSub{overwrite} {Specify if any previous file with the same name as the output file should be overwritten or appended to}
\par\textbf{\commandlabsubarg{report}{type}{layer\_derived\_view}}\par
\defSub{layer} {Define the label of the layer to print}
\defSub{years} {Define the years for the printing of the layer}
\defSub{time\_step} {Define the time-step for the printing of the layer}
\defSub{file\_name} {Define the name of the output file where the report is written}
\defSub{overwrite} {Specify if any previous file with the same name as the output file should be overwritten or appended to}
\par\textbf{\commandlabsubarg{report}{type}{selectivity}}\par
\defSub{selectivity} {Define the label of the selectivity to print}
\defSub{file\_name} {Define the name of the output file where the report is written}
\defSub{overwrite} {Specify if any previous file with the same name as the output file should be overwritten or appended to}
\par\textbf{\commandlabsubarg{report}{type}{random\_number\_seed}}\par
\defSub{file\_name} {Define the name of the output file where the report is written}
\defSub{overwrite} {Specify if any previous file with the same name as the output file should be overwritten or appended to}
\par\textbf{\commandlabsubarg{report}{type}{age\_size}}\par
\defSub{age\_size} {Define the label of the age-size relationship}
\defSub{file\_name} {Define the name of the output file where the report is written}
\defSub{overwrite} {Specify if any previous file with the same name as the output file should be overwritten or appended to}
\par\textbf{\commandlabsubarg{report}{type}{size\_weight}}\par
\defSub{age\_size} {Define the label of the age-size command that specifies the size-weight relationship}
\defSub{sizes} {Define the sizes for which to report the mean weights}
\defSub{file\_name} {Define the name of the output file where the report is written}
\defSub{overwrite} {Specify if any previous file with the same name as the output file should be overwritten or appended to}
\par\textbf{\commandlabsubarg{report}{type}{age\_weight}}\par
\defSub{age\_size} {Define the label of the age-size relationship}
\defSub{file\_name} {Define the name of the output file where the report is written}
\defSub{overwrite} {Specify if any previous file with the same name as the output file should be overwritten or appended to}
\par\textbf{\commandlabsubarg{report}{type}{MCMC}}\par
\defSub{file\_name} {Define the name of the output file where the report is written}
\defSub{overwrite} {Specify if any previous file with the same name as the output file should be overwritten or appended to}
\par\textbf{\commandlabsubarg{report}{type}{MCMC\_samples}}\par
\defSub{file\_name} {Define the name of the output file where the report is written}
\defSub{overwrite} {Specify if any previous file with the same name as the output file should be overwritten or appended to}
\par\textbf{\commandlabsubarg{report}{type}{MCMC\_objectives}}\par
\defSub{file\_name} {Define the name of the output file where the report is written}
\defSub{overwrite} {Specify if any previous file with the same name as the output file should be overwritten or appended to}
\subsection{Other commands and subcommands}\par
\defComArg{include}{file}{\I{Include an external file}}\par\par
